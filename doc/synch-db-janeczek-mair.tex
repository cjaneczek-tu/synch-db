\documentclass[11pt,a4paper]{article}
\usepackage[utf8]{inputenc}
\usepackage{amsmath}
\usepackage{amsfonts}
\usepackage{amssymb}
\usepackage{graphicx}
\usepackage{booktabs}
\usepackage[left=2cm,right=2cm,top=2cm,bottom=2cm]{geometry}
\usepackage{authblk}
\usepackage{fancyhdr}

\title{\bf DEZSYS03 - Synchronisation of heterogene databases}
\author{Janeczek Christian, Mair Wolfgang} 
\affil{IT Department TGM, Vienna}
\date{\today{}, Vienna}

\begin{document}


\maketitle
\pagestyle{fancy}
\fancyhf{}
\newpage
\tableofcontents
\rhead{Christian Janeczek, Wolfgang Mair}
\lhead{Synch-DB}
\chead{TGM 5AHITT}
\rfoot{Seite \thepage}
\newpage

\section{Task Description}
\textbf{\underline{DezSys03 - Synchronisation of heterogenen Datenbanken}}

\subsection{Introduction}

Dokumentieren Sie Ihren Versuch zwei heterogene Datenbanksysteme (MySQL, Postgresql) zu synchronisieren. Verwenden Sie dabei unterschiedliche Schemata (verschiedene Tabellenstruktur) und zeigen Sie auf, welche Schwierigkeiten bei den unterschiedlichen Heterogenitätsgraden auftreten können (wie im Unterricht besprochen) [2Pkt].

\noindent Implementieren Sie eigenständig eine geeignete Middleware [8Pkt]. Testen Sie Ihr gewähltes System mit mehr als einer Tabelle [4Pkt] (Synchronisation bei Einfügen, Ändern und Löschen von Einträgen) und dokumentieren Sie die Funktionsweise, sowie auch die Problematiken bzw. nicht abgedeckte Fälle [2Pkt].

\noindent Das PDF soll ausführlich beschreiben, welche Annahmen getroffen wurden. Der Source-Code muss den allgemeinen Richtlinien entsprechen und ebenfalls abgegeben werden.
\newline\newline
\noindent Gruppengröße: 2 
\newline
\noindent Gesamtpunkte: 16 [Aufteilung in eckigen Klammern ersichtlich]

\subsection{Rating}

Dokumentation der Synchronisation [2Pkt]

\noindent Implementierung der Middleware [8Pkt]

\noindent Zeittrigger bzw. Listener für Synchronisation bzw. DBMS Logs
Konfiguration bez. Mapping der Tabellen und Attribute
Konfliktlösung bei Zeitüberschneidung bzw. Datenproblemen (Log)
LostUpdate-Problem

\noindent Test mit mehr als einer Tabelle und mindestens 10 Datensätze pro Tabelle [4Pkt]

\noindent Uni- und Bidirektionale Änderungen mehrerer Tabellen
Einfügen, Ändern und Löschen

\noindent Dokumenation der Funktionsweise, Problematiken und Problemfälle [2Pkt]

\noindent Designdokumentation (Code + DB)
Synchronisationsverhalten
unbehandelte Problemfälle

\noindent Protokoll und Sourcecodedokumentation [0..-6Pkt]

\newpage
\section{Technology}
The simple but adventurous technology in this given task is known as the \textbf{Synchronization of Heterogenous Databases}. You might wonder: 'What exactly can I understand under the term Synchronization of Heterogenous Databases?' Come join me, while I explain everything to you.
\newline \newline
\noindent First of all, we have to decide how we are going to synchronize our two already existing databases. It was Bilbo Baggins' big goal, to finally see Elves again so he went straight to Rivendell. In our case, we have to choose our method of creating a bond between the two DBs. We have a slighty good disposal at our hands, so we decided to create a Middleware in Java, which should take on the job as Manager of our employees.
\newline \newline
\noindent That being said, if something should change at any of those two databases, may it be an INSERT to add some lovely new values to our tables, may it be an UPDATE to adjust/correct some mistakes we have done, or even if we have to fire one of our most trusted employees, the Middleware has got to create a layer of synchronicity. 

\subsection{Example}

\textbf{The following scenario:}\newline \newline
In the last months of summer there was built a bank and you have got too much money in your purse.
Consider the thought of adding some dollars, euros, pounds, yeng to your bank account. Wouldn't it be tragic if those added numbers aren't going to reach all the other databases with your highly loved money? To synchronize those values to one sum, we are in desperate need of the technology known as: \textbf{The Synchronization of Heterogenous Databases}

\newpage
\section{Design Consideration}
\subsection{Creating an Entity-Relationship-Diagram}
The first task would be to create an Entity-Relationship-Diagram to begin with. From this point on out we are able to discuss our current design concept. The issues we are going to test with our ER-Diagram are:

\begin{itemize}
	\item The synchronisation of splitted tables\newline  (Person,Mitarbeiter)\textless--\textgreater  (Angestellter)
	\item The synchronisation with columns which exits only in one Database \newline (Mitarbeiter.anzjahre)\textless--\textgreater (none)
	\item The synchronisation of information which is splitted in one table \newline (Person.vName, Person.nName)\textless--\textgreater (Angestellter.name)
	\item The synchronisation of columns with different datatypes \newline (Mitarbeiter.mongehalt - FLOAT)\textless--\textgreater (Angestellter.gehalt - DECIMAL)
	\item The synchronisation of columns with different column-names \newline (Mitarbeiter.mongehalt)\textless--\textgreater (Angestellter.gehalt)
	\item The synchronisation of two columns with the same name \newline (Mitarbeiter.wohnort)\textless--\textgreater (Angestellter.wohnort)
	
\end{itemize}

 \noindent The next step would be to implement the whole idea into the already downloaded OracleXE Virtual Machine. But before that, the second task has to be fulfilled: you know it: Creating a Relational Model.
\subsection{Creating a Relational Model}
The creation of the Relational-Model had to be done simultaneously to the creation of the ER-Diagram to save as much time humanly possible. The underline command will be very useful here to differ between Primary Keys and Foreign Keys. Creating the Relational Model will be an easy task, because we have direct intercommunication with each person working on this very application.

\subsection{Creating the specific DBMS-varying CREATE scripts}
To fulfill the requirements, we have to somehow synchronize our two databases, which are located in two different Database Management Systems. But before that, we have to create our tables, which are then to be matched by either:
\begin{itemize}
	\item Full Extract and Delta
	\item Trigger
	\item Logging files
	\item ...
\end{itemize}
\noindent The created databases look like the following: \newline \newline
\textbf{MySQL} Abteilung(id:INT, name:VARCHAR, PK(id)) , Angestellter(id:INT, name:VARCHAR, gehalt:DECIMAL, anzjahre:INT, wohnort:VARCHAR, abteilung:INT, PK(id), FK(Abteilung.id)) \newline \newline
\textbf{postgreSQL} Person(id:SERIAL, vName:VARCHAR, nName:VARCHAR, wohnort:VARCHAR, PK(id)) , Mitarbeiter(id:SERIAL, mongehalt:FLOAT, idPerson:INT, PK(id), FK(Person.id))

\newpage
\section{Apportionment of work with effort estimation}

\textbf{Janeczek:} \\ \\
\textbf{Mair:} \\ \\

\newpage
\section{Task Execution}
\subsection{Example}

\newpage


\end{document}
