\documentclass[11pt,a4paper]{article}
\usepackage[utf8]{inputenc}
\usepackage{amsmath}
\usepackage{amsfonts}
\usepackage{amssymb}
\usepackage{graphicx}
\usepackage{booktabs}
\usepackage[left=2cm,right=2cm,top=2cm,bottom=2cm]{geometry}
\usepackage{authblk}
\usepackage{fancyhdr}

\title{\bf DEZSYS03 - Synchronisation of heterogene databases}
\author{Janeczek Christian, Mair Wolfgang} 
\affil{IT Department TGM, Vienna}
\date{\today{}, Vienna}

\begin{document}


\maketitle
\pagestyle{fancy}
\fancyhf{}
\newpage
\tableofcontents
\rhead{Christian Janeczek, Wolfgang Mair}
\lhead{Synch-DB}
\chead{TGM 5AHITT}
\rfoot{Seite \thepage}
\newpage

\section{Task Description}
\textbf{\underline{DezSys03 - Synchronisation of heterogenen Datenbanken}}

\subsection{Introduction}

Dokumentieren Sie Ihren Versuch zwei heterogene Datenbanksysteme (MySQL, Postgresql) zu synchronisieren. Verwenden Sie dabei unterschiedliche Schemata (verschiedene Tabellenstruktur) und zeigen Sie auf, welche Schwierigkeiten bei den unterschiedlichen Heterogenitätsgraden auftreten können (wie im Unterricht besprochen) [2Pkt].

Implementieren Sie eigenständig eine geeignete Middleware [8Pkt]. Testen Sie Ihr gewähltes System mit mehr als einer Tabelle [4Pkt] (Synchronisation bei Einfügen, Ändern und Löschen von Einträgen) und dokumentieren Sie die Funktionsweise, sowie auch die Problematiken bzw. nicht abgedeckte Fälle [2Pkt].

Das PDF soll ausführlich beschreiben, welche Annahmen getroffen wurden. Der Source-Code muss den allgemeinen Richtlinien entsprechen und ebenfalls abgegeben werden.

Gruppengröße: 2
Gesamtpunkte: 16 [Aufteilung in eckigen Klammern ersichtlich]

\subsection{Rating}

Dokumentation der Synchronisation [2Pkt]

Implementierung der Middleware [8Pkt]

Zeittrigger bzw. Listener für Synchronisation bzw. DBMS Logs
Konfiguration bez. Mapping der Tabellen und Attribute
Konfliktlösung bei Zeitüberschneidung bzw. Datenproblemen (Log)
LostUpdate-Problem

Test mit mehr als einer Tabelle und mindestens 10 Datensätze pro Tabelle [4Pkt]

Uni- und Bidirektionale Änderungen mehrerer Tabellen
Einfügen, Ändern und Löschen

Dokumenation der Funktionsweise, Problematiken und Problemfälle [2Pkt]

Designdokumentation (Code + DB)
Synchronisationsverhalten
unbehandelte Problemfälle

Protokoll und Sourcecodedokumentation [0..-6Pkt]

\newpage
\section{Technology}
\subsection{Example}


\newpage
\section{Design Consideration}
\subsection{Creating an Entity-Relationship-Diagram}
The first task would be to create an Entity-Relationship-Diagram to begin with. From this point on out we are able to discuss our current design concept. The Issues we are going to test with our ER-Diagram are:

\begin{itemize}
	\item The synchronisation of splitted tables (Person,Mitarbeiter) <--> (Angestellter)
	\item The synchronisation with columns which exits only in one Database (Mitarbeiter.jahranz) <--> /
	\item The synchronisation of information which is splitted in one table (Person.vName, Person.nName) <--> (Angestellter.name)
	
\end{itemize}

 The next step would be to implement the whole idea into the already downloaded OracleXE Virtual Machine. But before that, the second task has to be fulfilled: you know it: Creating a Relational Model.
\subsection{Creating a Relational Model}
The creation of the Relational-Model had to be done simultaneously to the creation of the ER-Diagram to save as much time humanly possible. The underline command will be very useful here to differ between Primary Keys and Foreign Keys. Creating the Relational Model will be an easy task, because we have direct intercommunication with each person working on this very application.

\newpage
\section{Apportionment of work with effort estimation}

\textbf{Janeczek:} \\ \\
\textbf{Mair:} \\ \\

\newpage
\section{Task Execution}
\subsection{Example}

\newpage


\end{document}
